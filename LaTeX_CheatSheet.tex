\documentclass[font size]{type of document}
Ex: \documentclass[11pt,a4paper]{article}

\begin{document} : Before beginning any LaTeX document
\end{document}: To end any LaTeX document

Write all the content between begin and end document tags.

% : To comment out something

\parindent 0px : It will not ident the paragraphs in the document

\pagestyle{empty}: To remove the page number

\LaTeX\ : To get the word LaTeX in special font

$Math expression$ : To write any mathematical expression

${Math Expression}$ : To bring the whole mathematical expression on one line

$$Math Expression$$ : To write mathematical expression and print on other line separately

\ldots: Brings '...' in the document

Greek Letters:

$$\pi$$ : prints lowercase π

$$\Pi$$ : print uppercase Π

$$\<GreekLetterName>$$: prints the greek letter

$$A= \pi r^2$$: prints A = πr2

Trigonometric functions:

$$y= \sin x$$: Prints y=sinx

$$y= \sin^{-1} x$$: Prints sin invers x

Logarithmic Functions:

$$y= \log x$$ : Prints y=logx

Root Function:

$$\sqrt(number)$$ : This will give square root of the number

$$ \sqrt[n]{x} $$ : This will print nth root of x

Fractions:

$$ \frac{numerator}{denominator}$$

Brackets:

\usepackage{amsfonts,amssymb, amsmath} : To get some special symbols, fonts [Use this just after \documentclass tag]

\in : To get the 'belongs to' symbols ∈

\mathbb{R} : To get the 'Real Number' Symbol. 

\\[6pt] : It will create a linked break of width 6 pt. You can change the number.

$\{content\}$ :  It will not ignore the curly bracket then

$\$$: It will print the dollar symbol

$$\left({\frac{n}{d}\right)$$ : This will make the size of () according to the size of equation

Sample Code for different types of bracket:
$$\left(\frac{1}{\sqrt{x^2+y^2}}\right)$$
$$\left[\frac{1}{\sqrt{x^2+y^2}}\right]$$
$$\left\{\frac{1}{\sqrt{x^2+y^2}}\right\}$$
$$\left \langle   \frac{1}{\sqrt{x^2+y^2}}  \right \rangle$$
$$\left | \frac{1}{\sqrt{x^2+y^2}}\right |$$
$$\left. \frac{dy}{dx} \right |_{x=1}$$ 

Tables:

\begin{tabular}{cccccc}
--data with & sign in between--
\end{tabular} : It will create 6 columns in which data will be centred alinged

\begin{tabular}{llllll}
--data with & sign in between--
\end{tabular} : It will create 6 columns in which data will be left alinged

\begin{tabular}{rrrrrr}
--data with & sign in between--
\end{tabular} : It will create 6 columns in which data will be right alinged

\begin{tabular}{ccrrll}
--data with & sign in between--
\end{tabular} : It will create 2 columns in which data will be centred alinged, 2 columns right aligned and 2 columns left aligned.

\begin{tabular}{|c|c|c|c|c|c|}
first row data with & sign in between\\ \hline [to bring horizontal line]
second row data with & sign in between\\
third row data with & sign in between\\
\end{tabular} : It will create 6 columns with vertical bars in between in which data will be centred alinged

\begin{tabular}{|c|p{2cm}|}
--data with & sign in between--
\end{tabular} : It will create 2 columns in which data will be one will be centred alinged, and one will be a paragrapgh.

Sample Code:

\begin{table}[H]
\centering
\def\arraystretch{1.5}
\begin{tabular}{|c||c|c|c|c|c|} \hline
$x$ & 1 & 2 & 3 & 4 & 5 \\ \hline
$f(x)$ & 10 & 20 & 30 & 40 & 50\\ \hline
\end{tabular}
\caption{These values represent}
\end{table}

Arrays:
/begin{align}
Your text
/end{align}

Lists:

\begin(enumerate)
\item itemname
\item itemname
\end(enumerate) : This will create a numbered list

\begin(itemize)
\item itemname
\item itemname
\end(itemize) : This will create a bulleted list

\begin(itemize)
\item itemname
\item itemname
	\begin(itemize)
	\item itemname
	\item itemname
	\end(itemize)
\end(itemize) : This will create a nested list


Text Document Formatting:

\textit{your text} : This will write the text btwn brackets in Italics

\textbf{your tex} : This will write the text btwn the brackets in Bold.

\textsc{your text} : This will write the text in small caps.

\texttt{your text} : This will write the text in typewriter font.

\usepackage{hyperef} : This will make the link clickable in the document

\url{your link} : This will make the link clickable and will be written in typewriter format

\href{your link}{Text} : When we click on the text it will redirect to the link

\begin{large}Your Text\end{large} : The written in between will be written in large


Sample Code:
\vspace{1cm}
\begin{large}Your Text\end{large}\\
\vspace{1cm}
\begin{Large}Your Text\end{Large}\\
\vspace{1cm}
\begin{huge}Your Text\end{huge}\\
\vspace{1cm}
\begin{Huge}Your Text\end{Huge}\\

\begin{center}
Your Text
\end{center} : Text written in between will be centred

\begin{flushleft}
Your Text
\end{flushleft} : Text written in between will be left

\begin{flushright}
Your Text
\end{flushright} : Text written in between will be right

\title{text}
\author{text}
\date{date} : Write all these before \begin{document} 
Then to print all of this site \maketitle 

 
\date{\today} : It will take the date on which it was last compiled.

Sections:

\section{Section Name} : it will create a section

\subsection{SubSection name} : after the section tag go on next line give a intend and then use to create subsections

\subsubsection{name} : It will create subsubsection

\section*{section Name} : It will create a section without any numbering 

\tableofcontents : It will create a new page and add all sections heading under it

Packages:

\usepackage{fullpage} : It will create a 1inch margin on all sides

\usepackage[top= 1in, bottom= 1in, left=0.5in , right=0.5in]{geometry}: Fully customizable margin

\includegraphics[height= ]{image name} : Image must be in same folder as your .tex file
Macros:

\def\eq1{your eqn} : So whenever you type $eq1$ it will print the feeder eqn.

Calculus Notation:

$\displaystyle{\lim\limits_{x \to a}$ : To show limit

$\int_a^b$ : It will show integration symbol with limits a to b
